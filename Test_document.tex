\documentclass{yannick}

	%%%%%%%%%%%%%%%%
	% Titlepage %%%%
	%%%%%%%%%%%%%%%%

\title{LaTeX template for scripts } 
\author{by Yannick Kees \\  {\small University Bonn}}
\date{Summer 2021 }

	
	%%%%%%%%%%%%%%%%
	% Pagestyle %%%%
	%%%%%%%%%%%%%%%%	
	
\pagestyle{fancy}
\rhead{ \includegraphics[scale=0.03]{unibonn.png}}
\lhead{\textsc{Template file - Yannick Kees}}
\cfoot{\thepage }


\begin{document}

\maketitle
\newpage

	\section{Examples for using this class}
	% You can use 'dfn', 'satz', 'satzb', 'satzn', 'satznb', 'lem', 'lemn', 'lemnb', 'lemb', 'korollar'
	% 'yannick' & 'problem'
	\begin{dfn}
		We define the \textbf{Laplace-operator} as
			\begin{align*}
				\Delta f = \sum_{i=1}^n \frac{\partial^2 f}{\partial x_i^2}
			\end{align*}
	\end{dfn}

	\begin{satznb}{Fundamentallemma of calcus of variations}
		Let $\Omega$ be an open subset of $\mathbb{R}^2$ and $\Phi:\Omega\to\mathbb{R}$ be lokal integrable. If for any function $v:\Omega\to\mathbb{R}$ with compact support the integral 
		\begin{align*}
			\int\limits_{\Omega} \Phi(x)v(x)\,\mathrm{d}x
		\end{align*}
		vanishes, then $\Phi(x) = 0$ nearly everywhere.
	\end{satznb}

	\begin{satzn}{First green identity}
		{
			Let $\omega\subset \mathbb{R}^n$ be compact and $\phi$ and $\psi$ are two functions on $\Omega$, where $\phi$ is once and $\psi$ is twice differentable. Then
			\begin{align*}
				\int\limits_\Omega \phi \Delta(\psi)+\nabla\phi\cdot\nabla\psi\,\mathrm{d}m^d = \int\limits_{\partial \Omega}\phi \frac{\partial \psi}{\partial n}\,\mathrm{d}m^{d-1}
			\end{align*}
		}
		We use the gaußsche Integral formula to see that
		\begin{align*}
		 	\int\limits_{\partial \Omega}\phi \frac{\partial \psi}{\partial n}\,\mathrm{d}m^{d-1}&=  \int\limits_{\partial \Omega}(\phi\nabla\psi)\cdot \vec{n}\,\mathrm{d}m^{d-1}\\
		 	&= \int\limits_\Omega \nabla\mathrm{div}(\phi\nabla\psi)\\
		 	&= \int\limits_\Omega \phi \Delta(\psi)+\nabla\phi\cdot\nabla\psi\,\mathrm{d}m^d
		\end{align*}
	\end{satzn}
	
	\begin{korollar}
		If $\mathcal{L}u\geq 0$ in $\Omega$, then $u$ obtains its minimum in the boundary of $\Omega$.
	\end{korollar} 
	
	\begin{dfn}
		A problem is called \textbf{well posed}\index{well posed} if there exists a solution, that is unique and depends continuously on its data.
	\end{dfn}

	\begin{problem}{Laplace-Equation}{
		\begin{align*}
			\Delta u(x)&=0 && \text{for }x\in\Omega \\
			u(x)&=g(x) &&\text{for }x\in \Gamma
		\end{align*}
	}\end{problem}

	\begin{lem}
		{
			 Stability implies $\|A^{-1}_h\|_{\infty }\leq C_s$ and this bound is independent of $h$.
		}
		Let $\vec{v}_h$ be the coefficient vector of $v_h$ and $\vec{w}:= A_h\vec{v}_h $ is the coefficient vector of $\mathcal{L}_hv_h$
		\begin{align*}
			\|A^{-1}_h\vec{w}_h\|_{\infty }=\|\vec{v}_h \|_{\infty}=\|v_h \|_{\Omega}\leq C_s \|\mathcal{L}_hv_h \|_{\overline{\Omega_h}} = C_s\|A_h\vec{v}_h \|_{\infty}=C_s\|\vec{w}_h \|_{\infty}
		\end{align*}			 
		Then $\|A^{-1}_h\|_{\infty }$ is the smallest number for which this inequality holds.
	\end{lem}
		 
	\begin{yannick}{
		In contrast to Dirichlet boundary conditions, Neumann boundary conditions are not directly built into the search space. Therefore they are also calles \textbf{natural boundary conditions}.
	}\end{yannick}	
	
	\begin{lemn}{Comparison Principle}
		{ 
			If $\mathcal{L}u\leq\mathcal{L}v$ in $\Omega$ and $u\leq v$ in the boundary, then $u\leq v$ in $\overline{\Omega}$.
		}
		We use the maximum principle for
		\begin{align*}
			\mathcal{L}(v-u)\leq 0
		\end{align*}
		Then $(v-u)\leq 0$ in $\overline{\Omega}$.
	\end{lemn}
	
	\begin{lemb}
		Let $\mathcal{L}$ be uniformly elliptic. Then there exists a constant $c:=c(\Omega,\alpha)$, such that
		\begin{align*}
			|u(\vec{x})|\leq \max_{\vec{z}\in\Gamma}|u(\vec{z})|+c\cdot\sup_{\vec{z}\in\Omega}|(\mathcal{L}u )(\vec{z})|   &&\forall u\in C^2(\Omega)\cap C(\overline{\Omega}),\ \vec{x}\in\Omega
		\end{align*}
	\end{lemb}
	
	\begin{satz}
		{
	 		The space $C^\infty$ is dense in $H^m(\Omega)$.
	 	}
	 This theorem was proofen in 1964 by Meyers and Serrin. ($H=W$).
	 \end{satz}
\end{document}
